\documentclass[
  article,
  12pt,
  openany,
  oneside,
  a4paper,
  chapter=TITLE,
  hyphens,
  english,
  brazil,
  sumario=tradicional
]{abntex2}

\usepackage[T1]{fontenc}
\usepackage{times}
\usepackage[utf8]{inputenc}
\usepackage{color}
\usepackage{microtype}
\usepackage{titlesec}
\usepackage[brazilian,hyperpageref]{backref}
\usepackage[alf,abnt-emphasize=bf,abnt-etal-cite=3]{abntex2cite}
\usepackage{indentfirst}
\setcounter{chapter}{1}

\usepackage{lipsum}

% Citações
%\renewcommand{\backrefpagesname}{Citado na(s) página(s):~}
%\renewcommand{\backref}{}
%\renewcommand*{\backrefalt}[4]{}
%  \ifcase #1 %
%  Nenhuma citação no texto.%
%  \or
%  Citado na página #2.%
%  \else
%  Citado #1 vezes nas páginas #2.%
%  \fi}%
 
% Espaçamentos em geral
\setlength{\parindent}{1.5cm}
%\setlength{\parskip}{0.2cm}
%\onelineskip
\setlrmarginsandblock{3cm}{2cm}{*}
\setulmarginsandblock{2.5cm}{2.5cm}{*}
\checkandfixthelayout

% Metadados do documento
\titulo{Trabalho Final de Treinamento\\BlackJack}
\autor{Lucas Samuel Vieira}
\local{Diamantina}
\data{2018}
%\instituicao{NextStep}
\tipotrabalho{Instruções}
\preambulo{Instruções do trabalho final de treinamento da NextStep.}

\definecolor{blue}{RGB}{41,5,195}
\makeatletter
\hypersetup{
  %pagebackref=true,
  pdftitle={\@title},
  pdfauthor={\@author},
  pdfsubject={Instruções do trabalho final de treinamento da NextStep.},
  pdfcreator={LaTeX with abnTeX2},
%  pdfkeywords={abnt}{latex}{abntex}{abntex2}{atigo científico},
  colorlinks=true,                % false: boxed links; true: colored links
  linkcolor=black,                 % color of internal links
  citecolor=black,                 % color of links to bibliography
  filecolor=black,                      % color of file links
  urlcolor=black,
  bookmarksdepth=4
}
\makeatother


\makeindex



\begin{document}
\OnehalfSpacing
\pretextual
%\maketitle

\imprimirfolhaderosto

\textual

\section{Regras do Jogo}

O jogo a ser implementado será o jogo de cartas Blackjack, também conhecido como Vinte-e-um, que consiste em utilizar-se da sorte para vencer.

Podem participar do jogo de duas a sete pessoas, e vence o jogador que, ao somar os valores das cartas que possui, chegar mais próximo do número 21, mas sem ultrapassá-lo. O jogador que ultrapassar o número perde e, caso dois ou mais jogadores terminem a partida com o exato valor de soma, estes deverão efetuar uma rodada de desempate.

Ao início de cada rodada, cada jogador recebe duas cartas, e poderá opcionalmente pedir à banca que lhe dê mais uma carta, até que possua uma soma satisfatória, ou até que a soma passe do número vinte-e-um. Quando todos os jogadores estiverem satisfeitos, as cartas são todas abaixadas, e os jogadores, então, saberão se algum deles venceu, ou se deverá ser feita uma rodada de desempate.

\section{Detalhes Planejados de Implementação}

\begin{itemize}
\item O jogo será implementado utilizando o framework CSS Materialize;
\item A versão eletrônica do jogo suportará apenas um jogador, e mais três jogadores controlados por IA;
\item O jogador não poderá ver diretamente as cartas dos outros jogadores, a não ser ao fim da partida;
\item Cartas categorizadas como ``família real'' (valetes, damas, reis) serão contabilizados com o valor 10.
\end{itemize}

\end{document}
